\documentclass{article}		%Petite taille de doc
\usepackage{fullpage}
\usepackage[francais]{babel}	%Doc fr
\usepackage{todonotes}
\usepackage{amsfonts}
\usepackage{amsthm}
\usepackage{amsmath}
\usepackage{euler}
\usepackage{fontspec}
\setmainfont[Ligatures={Historic},Numbers=OldStyle]{Linux Libertine O}

\newtheorem{defi}{Définition}
\theoremstyle{definition}
\newtheorem{theo}{Théorème}
\theoremstyle{plain}
\begin{document}
\title{Learning Parities with Noise}
\author{Thomas Bourgeat }
\date\today
\maketitle
\section{Introduction au problème du LPN}
De plus en plus d'applications cryptographiques ont besoin d'être embarquées 
dans des dispositifs très économiques, voir jetables.  Le LPN est un candidat
pour permettre de faire de la cryptograpie, classiquement de
l'authentification, avec une puissance de calcul
très limitée et une diversité d'opérations très réduite.
Le standard AES est classiquement utilisé pour sa relative sécurité,
cependant les implémentations sont assez couteuses, et si le LPN s'averait suffisament
robuste, il serait une bonne alternative.
\\\\
Une fois un "bon" problème candidat à une application cryptographique, il
faut construire des protocoles pour être capable de lui faire faire de la
signature, du chiffrement ou de l'authentification. Ces protocoles ne
doivent pas introduire de faiblesses dans le système utilisé : ils doivent
être aussi difficile à casser que le problème initialement choisit est
difficile à résoudre.
Dans la suite nous  étudierons d'abord de l'authentification et du
chiffrement. Enfin, nous donnerons une idée de la difficulté du
problème du LPN par une vision des algorithmes actuels et plusieurs
réductions à d'autres problèmes connus.    
\subsection{Définition}
Dans la suite on travaillera, sauf mention explicite du contraire, dans
$GF(2)^n$, avec n la taille de clef. On note $Ber(\epsilon)$
une loi de Bernoulli de paramètre $\epsilon$ : qui vaut 1 avec
probabilité $1-\epsilon$. On rappelle que $\epsilon$ est petit : c'est à
dire $\epsilon < \frac {1}{2}$.  
\begin{defi}
Soit $m \in \mathbb{N}, A \in \mathcal{M}_{m,n}(GF(2)), s \in GF(2)^n$ et
$\nu \in GF(2)^m$ tel que $\nu_i \leftarrow Ber(\epsilon)$. On appelle
problème du LPN le fait de retrouver $s$ à partir de $A,\epsilon,
A.s+\nu$. 
\end{defi}
Géométriquement, on dispose de $m$ projections légèrement bruitées d'un vecteur secret sur m
vecteurs données, et on veut retrouver le vecteur secret.
\subsection{Spécificité du problème}
On commence par remarquer que si le bruit est nul, il est très facile de
résoudre le problème par un algorithme d'élimination de Gauss-Jordan.
Donc certaines instances sont très faciles. De même un structure
excessive sur la matrice A rend le problème simple : si chaque vecteur de
la base canonique est mis en ligne dans A, chacun un nombre de fois
suffisant, on peut alors résoudre le problème par un simple vote à la 
majorité.
\\
Pour faire un parallèle on peut penser au problème de factorisation d'un
nombre comme produit de deux premiers : la difficulté d'une instance à
l'autre varie beaucoup, ainsi il n'est pas immédiat de choisir une instance
que l'on sait difficile à résoudre. 
\\
On verra dans la suite que le LPN est relié aux problèmes de vecteurs
courts dans les réseaux ainsi qu'au problème de decodage. \cite{Alekh}
propose un cadre pour voir autour de quels problèmes certainement
difficile gravite le LPN, notamment il propose un cryptosystème autour du
lpn qui est intractable sous l'hypothèse d'une conjecture assez
plausible. Cependant on l'aura compris, on ne sait pas si le LPN est
difficile.

\section{Protocoles et preuves de sécurités}
\subsection{Chiffrement}
\subsubsection{Asymétrique}
\cite{Alekh} faisabilité : \cite{Dam} 
\subsubsection{Symétrique}

\subsection{Authentification clef publique : contribution }
\subsection{Authentification clef privé}
\subsubsection{HB}
\paragraph{Sécurisé dans le modèle passif}
\paragraph{Une attaque active}
\subsubsection{HB sharp}
\paragraph{Sécurisé dans le modèle actif}
\paragraph{Une attaque man-in-the-middle}
\subsubsection{Man-in-the-middle secure authentication schemes}
\paragraph{Sécurisé dans le modèle man-in-the-middle}
\paragraph{Une attaque concurrente}
\paragraph{TODO: Réfléchir à une proposition de correction}

\section{Modèles et cryptanalyses}
Les protocoles présentés précedemment font tous l'hypothèse que le
problème du LPN est difficile. Pour vérifier la solidité de cette
hypothèse,   nous allons
présenter rapidement les deux grands modèles d'apprentissage SQ et PAC, ainsi que le
résultat majeur de BKW de difficulté du LPN dans le modèle SQ. Ensuite
nous verrons les algorithmes classiques, des paramètres caractéristiques,
et des idées d'algorithmes ou d'améliorations testées durant le stage.

\subsection{Modèles}
Cette partie est inspirée des cours de Avrim
Blum à la Carnegie Mellon University
\subsubsection{Modèle PAC}
Le modèle PAC, pour Probably Approximately Correct, est un modèle
d'apprentissage ici présenté dans le cadre de la
classification : dans le cadre du LPN, la classe d'un vecteur $a$ est
$a.s$.
\begin{defi}
On dit qu'une classe de fonction $\mathcal{C}$ booléennes de l'hypercube
de dimension n est PAC-learnable si pour tout $i\in\mathcal{C}$, étant donné un oracle qui génère des
entrées
selon une loi $\mathcal{D}$ et qui donne une classification correct avec une
certaine probabilité $1-\eta$, $\forall
\delta<\frac{1}{2},\epsilon<\frac{1}{2}$ il existe un algorithme
$\mathcal{A}$ tel qu'avec probabilité au moins $1-\delta$ l'algorithme donne une
$\epsilon$-approximation de la fonction cible sur $\mathcal{D}$. 
\end{defi}
\subsubsection{Modèle SQ}
Le modèle SQ (pour Statistical Queries) propose un cadre pour étudier
certains problèmes d'apprentissages. Le modèle se définit comme une
interface restreinte entre l'algorithme d'apprentissage et la source de
données.
\\
\\
On dispose d'un oracle pouvant répondre à des questions de type "Quelle
est la probabilité que l'instance x vérifie le prédicat P". L'oracle
répond un intervalle de probabilité : c'est à dire "C'est entre
$p - \tau$ et $p + \tau$". La question posée doit-être évaluable
en temps polynomial.
\\
\begin{defi}
On dit qu'une classe de fonction $\mathcal{C}$ booléennes de l'hypercube
de dimension n est SQ-learnable sur la
distribution $\mathcal{D}$ si il y a un algorithme $\mathcal{A}$
tel que $\forall c \in \mathcal{C}, \epsilon>0$, $\mathcal{A}$ donne une
$\epsilon$-approximation de c à partir de l'oracle SQ. Le temps
d'execution d'algorithme, le nombre de queries et l'inverse de la plus
petite tolérance $\tau$ doivent être polynomiaux en $n$ et en
$\frac{1}{\epsilon}$. 
\end{defi}
\begin{defi}
On dit qu'une classe de fonction est faiblement SQ-learnable, s'il existe
un $\epsilon<\frac{1}{2}-\frac{1}{poly(n)}$ tel que la définition
précèdente soit vérifiée.(Quantification existentielle au lieu
d'universelle sur $\epsilon$) 
\end{defi}
Ce modèle, est en fait expressif, de nombreux problèmes sont
"SQ-learnable" comme on va le voir immédiatement.
\paragraph{Exemple de problèmes SQ-learnable}
Soit $f$ une fonction de l'ensemble des fonctions qui sont des
disjonctions de n variables booléennes. Par exemple
$f(x)=x_1 \vee x_4 \vee \dots \vee x_{n-2}$.
On cherche à l'apprendre. Pour ça on va faire les queries
$P(f(x)=0\wedge x_i=1)$ pour tout $i\in [1,n]$. Si la i-ème réponse est
suffisament petite : plus petite que $\frac{\epsilon}{n}$, alors $x_i$
n'est pas dans l'expression de f.  
\paragraph{SQ-learnable $\Rightarrow$ PAC-learnable}
Etant donné un nombre important de samples, on peut simuler l'oracle. Il
faut que la taille des samples soit grossièrement en
$O(\frac{1}{\tau^2})$, pour obtenir un précision de $\tau$.
  
\paragraph{Théorème de non SQ-learnabilité}
\cite{Blum} propose une caractérisation des problèmes SQ-learnable, que
l'on présente ici.
\begin{defi}
Soit $\mathcal{F}$ une classe de fonctions booléennes sur $GF(2)^n$, et $D$ une
distribution sur $GF(2)^n$.
On appelle dimension SQ de $\mathcal{F}$ le cardinal
du plus grand ensemble de fonctions $f_1,\dots, f_d$ telles que 
$$\forall i \not= j, |Pr_D(f_i=f_j)-Pr_D(f_i \not= f_j)|\leq \frac{1}{d^3} $$
\end{defi}
\begin{theo}
Soit $\mathcal{F}$ une classe de fonctions booléennes et $\mathcal{D}$
une distribution telle que $SQ_{dim}(\mathcal{F},\mathcal{D})\geq d \geq
16$. Alors si toutes les queries sont faites avec tolérance
$\tau\geq\frac{1}{d^{\frac{1}{3}}}$, au moins $d^{\frac{1}{3}}/2$ queries
sont necessaires pour apprendre $\mathcal{F}$ avec une erreur inférieur a
$\frac{1}{2}-\frac{1}{d^3}$ 
\end{theo}
\begin{theo}
L'ensemble des fonctions booléennes sur $GF(2)^n$ n'est pas
SQ-learnable.
\end{theo}
\begin{proof}
En effet la dimension SQ de l'ensemble des fonctions affines booléennes sur
$GF(2)^n$ n'est pas polynomiale :
\\\\
On prends $D$ la distribution uniforme, alors 
$\forall a,a'\in GF(2)^n,
Pr_D(a.x=a'.x)=\frac{1}{2}$,$Pr_D(a.x\not=a'.x)=\frac{1}{2}$
 De plus on ne peut pas faire des queries avec
des tolérances inférieures à celles de la proposition, en temps
polynomial.
\end{proof}
En conséquence ce dernier théorème nous donne une condition nécessaire
pour trouver des algorithmes en temps sous-exponentiel : Il faut qu'ils
utilisent des "opérations non SQ-like". Par exemple, des combinaisons
linéaires.

\subsection{Algorithmes pour attaquer le LPN}
Il est necessaire de distinguer deux grandes classes
d'algorithmes, en fonction du nombre de samples (donc d'authentification)
necessaire. Dans un premier temps nous allons étudier les algorithmes
necessitant peu de queries (donc plus lents), puis ensuite nous verrons
les algorithmes "BKW-like" necessitants plus de queries.


\subsubsection{Avec peu de queries}
\paragraph{Recherche exhaustive}
Comme toujours en cryptographie, on peut chercher à effectuer une attaque
par recherche exhaustive. Cependant ici plusieurs types de recherches sont
possibles.
\begin{itemize}
\item On peut effectuer une recherche exhaustive naive sur le
secret, et tester pour chaque secret si la proportion d'erreur est égale
au bruit.
\item On peut effectuer une recherche exhaustive sur le bruit, chercher
la position des erreurs, car on sait qu'il y a peu d'erreurs, ensuite
retrouver le secret associé par une élimination de Gauss, puis tester que
ce secret est en effet bon. Le problème est que l'élimination de Gauss
est assez couteuse : $O(n^3)$.
\item \cite{Gol} fait moralement quelque chose d'équivalent, mais
présenté de manière différente, ceci facilitant l'implémentation :
$s=A^{-1}.(\nu+b)$, et sa recherche
exhaustive sur s se fait en recherchant exhaustivement sur $\nu$ par poid
de hamming croissant. Cette manière d'énumérer permet de tomber beaucoup
%TODO: PRECISER
plus rapidement sur la solution. En pratique pour des bruits suffisament
faibles\footnote{0.01 pour n=768 par exemple}, c'est la meilleure attaque. Cette attaque a
le mérite de se paralléliser très facilement. 
\end{itemize} 

\paragraph{Attaque par vecteurs corrélés}
Le deuxième type d'attaque est asymptotiquement meilleur mais théorique.
L'idée consiste à profiter 
\subparagraph{Amélioration de \cite{Grigo}}
\begin{theo}\cite{Indyk}
      \\ Etant donné N points de l'hypercube de dimension d, on peut
trouver une paire de points $\delta$-proche, avec $\delta \leq
\epsilon.\rho$ où $\epsilon$ est la plus petite distance entre deux
 des N points. Ceci est fait en $O(d.N^{1+\frac{1}{\rho^2}+o(1)})$
        \end{theo} 
  L'approche consiste à supposer que le secret est sparse de poid de
hamming r
donné.\footnote{Plus tard on verra qu'en fait cette hypothèse est vide,
et qu'on peut réduire un problème LPN à un sparse LPN}
   L'idée de \cite{Grigo} est de construire tous les secrets de poid de
hamming $\frac{r}{2}$, d'évaluer 
      \begin{theo} Il existe un algorithme qui résoud un LPN r-sparse en
effectuant
$\frac{r.log(\frac{n}{\delta}).\omega(1)]}{\epsilon^2.(1-2.\eta)^2}$
queries en un temps
$$\frac{log(\frac{1}{\delta}}).n^{(1+(2.\eta)^2+o(1))r/2}{(1-2\eta)^2}$$
      \end{theo}
      \begin{proof}
      \end{proof}

\subparagraph{Amelioration de \cite{Valiant}} \cite{Valiant} propose une attaque qui utilise le gain de la
multiplication matricielle rapide.Son attaque ne fait pas intervenir le
bruit dans l'exposant

\paragraph{Attaque par dualité : Contribution }
Il est assez classique que le problème du LPN soit relié à un problème de
décodage : A est la matrice génératrice d'une code linéaire, et on
cherche à déterminer quel est le mot code réel $As$ en partant du mot code
bruité $As+\nu$, Autrement dit on cherche un vecteur court dans le noyau à
gauche de A.
\\
 Ici nous décrivons une autre manière de voir ce
problème, pour donner un sens à la matrice noyau à gauche de A. Cette
méthode s'est trouvée être a posteriori une réduction à un décodage de
code linéaire aléatoire.
\\\\
\label{SVP} 
On remarque que lorsque l'on xor deux équations, la validité de
l'équation xor est le xor de la validité des deux équations parentes.
Alors on va construire un système sur la validité des équations : en
effectuant des combinaisons qui lient les vecteurs de A, on va se ramener
à des équations vraies : $0=0$ et des équations fausses : $0=1$, et ces
équations sont issues d'un certain nombre de combinaison linéaires
d'équations initiales, on sait lesquels. Ces combinaisons linéaires sont en
fait le noyau à gauche de la matrice A, donc on peut en avoir une base en
temps polynomial que l'on empacte en ligne dans une matrice, que l'on
appelle $A^\dagger$. Bien entendu la dimension de cet espace est n,
cependant le LPN devient maintenant : trouver une solution courte au
système  $$A^\dagger.x=b$$ Autrement dit, trouver un nombre minimum
d'équations fausses qui justifierait que les équations produites de type $0=0$ et
$1=0$ soient respectivement vraies et fausses, \\\\
Pour simplifier le problème, on peut simplement garder des équations initiales dont le b vaut 0. 
L'avantage c'est qu'ainsi on cherche un
vecteur court dans $ker(A^\dagger)$. 


\subparagraph{Decoding in $2^{n/20}$} 
Le decoding problem possède une litterature très étoffée de \cite{Stern}
à \cite{Joux}. Il consiste à trouver le mot code à partir du syndrome (en
le calculant avec le mot code bruité). Les améliorations finales dues à
Joux et May donnent une complexité en $2^{n/20}$. Nous nous étendrons
pas sur ces méthodes parce qu'elles relèvent d'un autre domaine classique
et que nous ne pouvons pas facilement procéder à des tests (pas de
decoding implémenté en Sage).

\subparagraph{Via réduction de réseau} 
La recherche d'un vecteur court fait penser à la réduction de réseau et
donc à LLL. On sait en temps polynomial trouver d'assez bonnes bases dont
le premier vecteur court est optimal d'après \cite{Phong} à un facteur expérimentalement
$1.08^{\frac {n-1} 2}$ près.
Cependant LLL ne fonctionne pas dans $GF(2)$\footnote{les réseaux dans un
$GF(2)$ espace vectoriel étant moralement des sous-groupes de $GF(2)^n$
donc des $GF(2)^k$, ce n'est pas très riche, }. Néanmoins, remarquons que
si 
 $x\in GF(2)^n$, où l'espace est métrique pour la distance de hamming, considéré comme un vecteur de $\mathbb{R}^n$ on a :
$$HammingWeight(x)=\| x \| ^2$$  
 Martin Albrecht propose alors qu'en pratique un LLL dans R donnera un
vecteur court dont les composantes seront en valeur absolue plus petit
que 1, que l'on pourra ensuite re-projeter sur $GF(2)$ qui donnera un bon
vecteur court.
\\
On remarque dans tous les cas que $1.08^{\frac {n-1} {2}}$ devient
rapidement trop grand, et donc qu'LLL n'assure plus du tout de trouver un
vecteur optimal. Voir~\ref{Impsvp} pour des ordres de grandeurs.

\paragraph{Changement de modèle} Dans \cite{Arora} les auteurs constatent
que lorsque l'on change légèrement le modèle, on peut trouver des
solutions polynomiales. Plutôt que de tirer les samples au coup par coup,
ils proposent d'avoir un "bouton", qui permet d'obtenir m samples dont on
est sur qu'une fraction $\tau$ est bonne.
\todo[inline]{Ici}
\paragraph{Réduction de bruit : Contribution} \label{SAT} 
Jusqu'a présent nous n'avons pas encore vu de méthodes pour réduire le
bruit du LPN. Dans ce but on peut faire le OU de disons k équations, et
alors on obtient une équation juste avec très très grande probabilité.
Ainsi en augmentant le degrès des équations, on diminue le bruit à
vitesse exponentielle :
 $$P(\bar\nu_1 \vee \bar\nu_2 \vee \dots \vee \bar\nu_k)=(\epsilon)^k$$
\\\\
Cette stratégie génère donc un système d'équations polynomiales. Pour le
résoudre on peut envisager plusieurs solutions :
\begin{itemize}
\item Bases de Grobner
\item Relinéarisation : on pose tout les monomes présents comme des
variables
\item SAT-solving
\end{itemize}
Voir~\ref{ISAT} pour l'implémentation et les résultats.

\subsubsection{Approche "légo"}
Il s'avère que les algorithmes  necessitant un grand nombre
de queries sont tous construits autour des mêmes idées. Dans le but de
montrer cette unité, dans cette partie nous allons présenter différents
blocs de bases, extraits d'algorithmes divers. Ensuite nous décrirons les
différentes attaques comme des constructions avec ces blocs. Enfin nous
présenterons notre propre vision.  

\paragraph{Réduction de la taille du problème}
Le premier bloc consiste en la réduction de la taille du problème, au
prix d'une augmentation très significative du bruit.
\\
Pour celà, on prend nos samples, que l'on classe par groupes : sont dans
le même groupe tous les éléments qui ont en communs les $b$ premiers bits.
Ensuite dans chaque groupe on additionne un élément à tout les autres et
on supprime cet élément, voir figure~\ref{bbkw}. Ainsi tous les éléments ont leurs $b$ premiers
bits à zero. On recommence avec les $b$ suivant etc\dots
On répète cette opération a fois. Et on a ainsi réduit le problème à une
taille $n-ab$. En pratique pour faire les groupes on effectue simplement
un tri sur les bits concernés.
\begin{figure}
$$\begin{pmatrix}
\begin{cases}00\dots01 \\00\dots01\\ \vdots\\00\dots01 \end{cases} & \star \\
\begin{cases} 00\dots10 \\00\dots10\\ \vdots\\00\dots10  \end{cases} & \star  \\
\vdots & \vdots \\
\begin{cases} 11\dots11 \\11\dots11\\ \vdots\\11\dots11  \end{cases} & \star 
\end{pmatrix}$$
\caption{Réduction de taille par groupements, 1ère étape}
\label{bbkw}
\end{figure}
\\
\\
\\
\cite{BKW} à l'origine de cette idée montrent qu'alors le biais final est
de $(1-2\epsilon)^{2^a}$ et que la complexité en temps est de
$a(a2^b+\frac {n} {(1-2\epsilon)^{2^a}})$.

\paragraph{Réduction à un secret sparse.}
\cite{Kirchner}, \cite{Bernstein} remarquent que le problème du LPN peut
se réduire en temps polynomial a un problème du LPN avec un secret sparse
 par la suite d'opérations suivantes:
$$
\begin{cases}
A_1s+\nu_1=b_1\\
A_2s+\nu_2=b_1\\
\end{cases}
$$
$$
\begin{cases}
s=A_1^{-1}(b_1+\nu_1)\\
A_2A_1^{-1}\nu_1+\nu_2=b_2+A_2A_1^{-1}\nu_1
\end{cases}
$$

\paragraph{Stratégie de résolution finale}
\subparagraph{Vote}
On peut tout simplement espérer avoir les vecteurs de la base canonique
en ligne dans notre matrice, et espérer qu'on les a tous un nombre
suffisant de fois,
pour ensuite voter à la majorité, la valeur du bit de secret
correspondant. Il faut donc que le problème soit de petite dimension, et
qu'on ait beaucoup de samples. 
\subparagraph{FFT}
On sait que la transformée de Fourier de notre fonction, qu'on appelle
aussi transformée de Walsh-Hadamard, permet de trouver la meilleure
approximation affine d'une fonction booléenne.\\\\
 En effet : 
\begin{theo}
Les caractères irréductibles sont les $$\chi_y(x)=(-1)^{x.y},
\forall y \in GF(2)^n.$$
\end{theo}
\begin{theo}
Soit $\phi_y(x)=y.x=x_1.y_1+\dots+x_n.y_n$, $d$ la distance
de hamming, $\mathcal{W}$ la transformée de Walsh, et f une fonction
booléenne. Alors
$$d(f,\phi_y)=2^{n-1}-\frac{1}{2}.max\{\mid\mathcal{W}(x\mapsto(-1)^{f(x)})(y)\mid| 
y\in GF(2)^n \}$$
\end{theo}
\begin{proof}
On a : \\\\
\begin{tabular}{r c l}
$\mathcal{W}(x\mapsto(-1)^{f(x)})(y)$&$=$&$\sum\limits_{x\in GF(2)^n}(-1)^{f(x)+x.y}$\\
&$=$&$card\{x\in GF(2)^n | f(x)=x.y\}-card(x\in GF(2)^n | f(x)\not=x.y\}$\\
&$=$&$2^n-2d(f,\phi_y)$
\end{tabular}
\end{proof}
Donc il suffit de connaitre la position du coefficient de la transformée
le plus grand, pour connaitre la meilleure approximation affine, et donc
le secret. Cette procédure a été proposée par \cite{LF}.
En pratique le calcul d'une transformée de Walsh-Hadamard coûte $2^n.n$,
cependant \cite{Kirchner} propose de gagner un peu si le secret est
sparse. 
 
\subparagraph{Approche SFT}
\label{SFT}
Des algorithmes pour connaitre les coefficients de Fourier dominant d'une fonction d'un groupe
abélien ont été proposés par \cite{Mansour} (pour les fonctions booléennes)
puis  \cite{Akavia}.
\\
\cite{Vaudenay} décrit précisemment le fonctionnement de
l'algorithme, qui est une recherche dichotomique par le biais d'un
estimateur, que nous appellerons estimateur AKMV. Cet estimateur AKMV
permet d'estimer la somme des coefficients de Fourier qui sont dans un
voisinage (dont on peut controler la taille) d'un point donné. Ainsi
l'algorithme de SFT consiste simplement a faire une recherche
dichotomique, en éliminant à chaque fois les branches ne contenant pas un
poid en coefficient de Fourier suffisant.
\begin{defi}
 En dimension n, l'estimateur akmv en h de voisinage k vaut
$$akmv(h,k)=\frac{2^{2.n}}{N}.\sum_{i=0}^{n}
f(a_i+b_i).f(a_i+c_i).(-1)^{(b_i+c_i).h}$$
où 
\begin{itemize}
\item $a_i$ sont des mots quelquonques de n bits.
\item $N$ est indépendante de la dimension, permet de dimensionner la
finesse d'évaluation des coefficients.
\item $c_i,b_i$ sont des mots quelquonques de n bits possédants leur k derniers bits à
0.
\end{itemize}
\end{defi}

\begin{theo}\cite{Vaudenay}
$akmv(c,\delta)$ est un bon estimateur de la somme des carrés modules des coefficients
de fourier :
$$\sum_{\alpha\in I(c,\delta)} |\hat{f}(\alpha)|^2 - \gamma.2^{2.k}\leq
2^{2.k}.akmv(c,\delta)\leq
\sum_{\alpha\in I(c,\delta)} |\hat{f}(\alpha)|^2 +  \gamma.2^{2.k}$$
\begin{itemize}
\item $I(c,\delta)$ est l'ensemble des points qui possèdent les $\delta$
premières coordonnées de c, et des coordonnées suivantes quelquonques.
\item $\gamma=\frac{L_2(\hat{f})^2}{18.2.2^{2.k}}$
\item Probabilité d'echec plus petite que $2.\epsilon'$ 
\item $N=2.\gamma^{-2}.ln(\frac{1}{\epsilon'})$
\item Erreur bornée par $\frac{1}{18.2}.L_2(\hat{f})^2$ 
\end{itemize} 
\end{theo}

Dans le LPN on ne peut pas évaluer notre fonction n'importe où, donc on
ne peut pas effectuer l'algorithme de SFT \cite{Akavia}. Cependant on peut calculer
l'estimateur AKMV pour tous les points, pour un voisinage réduit au
centre.  Autrement dit on peut évaluer n'importe quel coefficient de
Fourier en $O(n)$. Donc pour nous : on peut tester un secret en $O(n)$.
Ce qui est mieux que le $O(n^2)$ traditionnel. \\
En fait on pourra critiquer aisément le fait que cette amélioration est
asymptotique, en pratique il y a un facteur entre $1000$ et $10000$ dans
le O\footnote{Selon la précision désirée dans l'approximation du
coefficient de Fourier}. 
\subsubsection{BKW et LF}
\begin{itemize}
\item \cite{BKW} part avec $2^\frac{n}{log(n)}$ samples, et effectue une
réduction de taille avec $a=log(n)$, $b = \frac{n}{log(n)}$.
 Puis ensuite effectue un vote à la majorité. La complexité
est de $O(2^{O(\frac {n} {log(n)})})$   
\item \cite{LF} commence de la même manière et termine par une
transformation de Walsh-Hadamard. La complexité est de
$O(2^\frac{n}{log(n)})$.
\end{itemize}
\subsubsection{Kirchner et Bernstein}
\cite{Bernstein}, \cite{Kirchner} proposent de faire toutes les étapes, et
choisir les paramètres de manières optimale : vu qu'ils gagnent un peu
dans la FFT lorsque le secret est très sparse, ils attendent que le
secret soit très sparse (par exemple moins de $4$ bits à $1$). Et pour
chaque essaie, ils effectuent quelques étapes de BKW (Pour ordre de
grandeur, Kirchner propose
pour $n=768$, $\epsilon=0.05$ de prendre $a=3$, $b=8$.)
\\Le principal avantage de cette approche est la diminution de la
quantité mémoire necessaire. Le nombre de queries ne diminue pas
beaucoup, mais le plongement dans une étape de randomization permet
d'économiser beaucoup de mémoire. C'est une amélioration significative
puisque c'était le principal bottleneck dans BKW, où la mémoire coûtait
aussi en $O(2^\frac{n}{log(n)})$ 
\\\\
Dans \cite{Bernstein}, ils propose de rajouter une structure
d'anneau : le ring-LPN. Ceci permettant de diminuer le coup des produits
matriciels.  

\subsubsection{Amélioration asymptotique par SFT : Contribution}
\cite{Vaudenay}
Si on garde la même trame que \cite{Kirchner} et \cite{Bernstein}, mais que l'on utilise notre
capacité à tester un secret en $O(n)$ vu dans~\ref{SFT}, on peut alors faire une phase
finale qui consiste à énumérer tous les mots de poid de hamming faible (les plus
probables), et de les tester. Ainsi, si le secret possède $w$ bits à $1$
dans ses $n-ab$ derniers bits,
la complexité de la phase finale est de $\binom{n-ab}{w}.n$, \footnote{Comme dans
\cite{LF}, \cite{BKW}, \cite{Kirchner}, \cite{Bernstein}, on ne s'attarde
pas à établir la complexité pour retrouver les $ab$ premiers bits.}

\paragraph{Optimisation des paramètres du précèdent.}
Donc si on prend notre dernier algorithme, sa complexité générale est
la suivante :
$$\underbrace{(\sum_{i=0}^{w}\binom{n-a.b}{i}.\epsilon^i.(1-\epsilon)^{n-a.b-i})^{-1}}_{\text{Attente
de secret w sparse}}.
(\underbrace{n^3}_{\text{Réduction sparse}} +
\underbrace{a.(a.2^b+\frac{n-a.b}{(1-2.\epsilon)^{2^a}})
}_{\text{Réduction BKW}} + \underbrace{\sum_{i=0}^w \binom{n-a.b}{i}.(n-a.b)}_{\text{Rétrouver solution par SFT}} )$$ 
Cette complexité est en fait simplifiée : elle est donnée ici en O, alors
que normalement il y a des constantes qui dépendent très fortement de
l'implémentation\footnote{Notamment des facteurs 1/32 ou 1/64 par
endroit si on fait des représentation compactes avec des entiers}. On sent
donc bien qu'il faut optimiser en a, b, w, à $\epsilon$ et n fixés. Pour
celà une étude analytique semble très délicate\footnote{Surtout qu'en
crypto les paramètres sont assez petits donc il ne fait aucun sens de
faire une étude asymptotique}. En pratique, l'espace des paramètres à
optimiser étant fini un petit script OCaml permet de donner les valeurs
optimales. 
\section{Implémentation}
\subsection{C}
\subsubsection{Éléments de bases pour LPN}
J'ai programmé en C un ensemble de fonctions pour pouvoir effectuer des
tests sur le LPN. L'implémentation est compacte : je stocke dans un
entier non-signé 32 valeurs binaires. Cette représentation compacte était
faite en prévision du coût mémoire des attaques BKW-like. Basé sur cette représentation, j'ai
programmé l'ensemble des outils de bases d'algèbre linéaire necessaires
simples :
\\
\begin{itemize}
\item Produit scalaire et matriciel naif 
\item Elimination de Gauss-Jordan
\item Ajout d'un $\epsilon$-bruit.
\item Génération des mots de poid de hamming fixé
\item Tri rapide sur les bits entre k et (k+i)
\item et de nombreux tests divers et variés comme l'étape de réduction de
BKW
\end{itemize}

\subsubsection{Phase finale du LPN en C}
J'ai implémenté en C, la phase finale qui consiste à faire une recherche
exhaustive en testant les mots de poid fixé, en $O(n)$. Via un calcul
d'approximation de coefficients de Fourier, expliquée en~\ref{SFT}.
\\
Cette phase est donc composée d'une recherche exhaustive sur des mots de
poid de hamming petit fixé. Pour paralleliser cette implémentation, j'ai
mis en bijection les mots de poid de hamming fixé et l'intervalle
d'entier $[0, C_{n}^{w}]$, de ce fait les itérations deviennent
entièrement indépendantes. Permettant une parallelisation à une première
échelle via MPI.
A une deuxième échelle et de manière transparente, j'ai utilisé openMP
pour paralleliser sur plusieurs coeurs, les boucles que doivent faire
chaque esclave. 
\\\\
Autre avantage : l'utilisation de MPI permet de broadcaster facilement
l'instance du LPN construite par le maitre à toutes les machines de la
grappe. En pratique le calcul a été fait sur 8 machines de
l'ens\footnote{The super-new spice machines}(2x Intel Xeon E5649 @
2.53GHz).  
\begin{figure}
\caption{Temps d'exécution en dimension n avec un secret k-sparse}
\begin{center}
\begin{tabular}{|c |c | c | c | c |c|c| }
\hline
   kn & 32 & 64 & 128 & 256 & 512 & 768 \\
\hline
 2  & &4 & 5 & 6 & &\\
\hline
  3 & &7 & 8 & 9 & &\\
\hline
 4 & & & & & & \\
 \hline
5 & & & & & & \\
 \hline
6 & & & & & & \\
\hline 
\end{tabular}
\end{center}
\end{figure}


\subsection{Sage : c'est le bien}
Une fois compilé (9h tout de même\dots), Sage est particulièrement efficace
pour tester rapidement des idées : je pense à un facteur de compression
sur la taille du code
entre 10 et 100. De plus Sage
implémente plus ou moins l'état de l'art des algorithmes classiques : multiplication
matricielle, SAT-solver etc\dots  


\subsubsection{Systèmes polynomiaux}\label{ISAT}
Le but fut d'implémenter ce que nous avons proposé en~\ref{SAT}
\\\\
En Sage, on déclare l'anneau dans lequel on travail, la méthode de
résolution du système et c'est fini. Autrement dit l'essentiel du travail
est de générer les équations : de traiter les données. Etant donné que la
complexité de SAT et Groëbner est très mal connue, le but était
uniquement d'effectuer un test, pour savoir si on se trouvait dans une
zone ou les méthodes usuelles se "comportent bien", ou pas\footnote{Je
présente ici chronologiquement ce que j'ai compris.}.
J'ai pris une instance du LPN de taille 32, générée en C, et ensuite j'ai
traité ces données via OCaml pour les formater pour Sage\footnote{Il
était possible de tout faire directement en Sage, mais je n'avais pas
fait de OCaml depuis plusieurs semaines!}. Les équations polynomiales
étaient toutes de degrès 2 : produit des équations $2.i$ et $2.i+1$
linéaire du problème LPN initial.
\paragraph{Résolution via Groëbner}
Exécuté sur ma machine plusieurs heures, ça n'a pas terminé. Peut-etre 
que le grand nombre de variables dans chaque équation est la cause de
l'echec de la résolution via
bases de Groëbner.
\paragraph{Résolution via SAT}
Via SAT, l'instance a été résolue en une dizaine de minutes. Ceci est
assez lent, à peine mieux qu'une recherche exhaustive. 
%TODO : PRECISER
\\
Après avoir regardé
un peu l'état de l'art du SAT solving, notamment \cite{Mezard}, il semble qu'une fois encore la
structure des entrées liées au LPN est très différente des données
classiques que l'on donne au SAT solver : la plupart du temps il y a un
nombre très réduit de littéraux dans chaque clauses, bien sur on peut se
ramener à ce cas en introduisant de nouvelles variables, mais alors le
rapport $\alpha=\frac {\sharp Clauses}{\sharp Variables}$ devient plus grand.
Or, \cite{Dubois} étudie que lorsque le nombre de variables tend vers
l'infini, presque toutes (au sens probabiliste) les
instances avec $\alpha > 4.506$ sont insatisfiables. Ce sont des
"hard-SAT"\footnote{Beaucoup de cycles dans le graphe de représentation}.

Ces lectures à posteriori\footnote{Toujours la difficulté de savoir s'il
est plus rapide d'implémenter pour tester, ou de trouver une biblio sur
le sujet} justifient que cette approche ne passera pas pour
des paramètres plus gros. 

\subsubsection{Réduction du LPN au Short Vector Problem}\label{Impsvp}
En \ref{SVP} on a vu une proposition de réduction du LPN au SVP dans un
sous-espace vectoriel du $\mathbb{Z}/2\mathbb{Z}$ espace vectoriel
$(\mathbb{Z}/2\mathbb{Z})^n$. L'implémentation choisie est celle avec une
réduction de réseau pour trouver le plus court vecteur. Cependant LLL
est fait pour des vecteurs dans réels, cependant on peut considérer les
vecteurs de $(\mathbb{Z}/2\mathbb{Z})^n$ comme des vecteurs réels,
effectuer LLL, qui ne fait que des combinaisons linéaires, puis se
replonger dans $GF(2)^n$.
\\ 
On remarque qu'il y a une relative corrélation entre un vecteur court de
$GF(2)^n$ avec la distance de hamming, et avec la norme 2. De plus
expérimentalement, LLL donne des vecteurs à valeur très majoritairement
dans $\{-1,0,1\}$, donc cette méthode\footnote{Piquée à Martin Albrecht}
est expérimentalement justifiée.
\\
Le principale bottleneck de cette méthode est le facteur d'approximation
de LLL, qui trouve un plus court vecteur à un facteur au plus 
$(1.08)^{\frac {n-1} {2}}$. Ainsi très rapidement lorsque la
dimension augmente, ou que le bruit augmente trop, LLL n'assure plus
de trouver
un vecteur suffisament court, et bien que l'algorithme soit
très rapide, il n'est plus correct.
En effet il faut que : $$1.08^{\frac {n-1} {2}}HammingWeight(\nu)\ll1/2
$$ 
 En pratique il faut alors l'éxecuter
de nombreuses fois pour trouver le bruit.On 
peut le voir dans le tableau de performance ci-dessous. Les tests sont
réalisés sur ma machine personnelle (AMD V120 @ 2.2GHz)
\\
\begin{figure}
\caption{Complexité en temps pour essayer une instance}
\begin{center}
\begin{tabular}{|c |c | c | c | c | }
\hline
   & 32 & 64 & 96 & 128 \\
\hline
 0.01  & &4 & 5 & 6 \\
\hline
  0.05 & &7 & 8 & 9 \\
\hline
 0.1& & & &\\
 \hline
 \end{tabular}
\end{center}
\end{figure}

\begin{figure}
\caption{Nombre d'instances necessaires avant réussite}
\begin{center}
\begin{tabular}{|c |c | c | c | c | }
\hline
   & 32 & 64 & 96 & 128 \\
\hline
 0.01  & &4 & 5 & 6 \\
\hline
  0.05 & &7 & 8 & 9 \\
\hline
 0.1 & & & & \\
 \hline
 \end{tabular}
\end{center}
\end{figure}

\bibliographystyle{alpha}
\bibliography{biblio}
\end{document}

